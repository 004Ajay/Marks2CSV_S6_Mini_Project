\chapter{Conclusion}
\section{Future Scope}

In conclusion, the Marks2CSV application has emerged as an invaluable solution for educational institutions in efficiently converting and processing students' marks. With its utilization of a meticulously developed Convolutional Neural Network (CNN) model, specifically tailored to the institution's requirements, the application has surpassed existing alternatives in terms of accuracy, precision, and recall. This accomplishment is a result of extensive research and the incorporation of modern techniques, positioning Marks2CSV at the forefront of marks data management.

One of the standout features of the Marks2CSV application is its exceptional efficiency. By delivering outputs in the form of machine-readable and writable CSV files within a matter of seconds, the application significantly accelerates data processing workflows. This expedited turnaround time translates into enhanced productivity, allowing educational institutions to swiftly access and analyze marks data, driving informed decision-making and timely interventions when necessary.


The Marks2CSV application has tremendous potential for further enhancements and expansion. Some of the future scopes include:
\begin{enumerate}
\item \textbf{Integration with Student Information Systems}: The application can be integrated with existing student information systems used by educational institutions, allowing seamless and automated extraction of marks data.

\item \textbf{Advanced-Data Analysis:} By incorporating advanced statistical analysis and data visualization capabilities, the application can enable deeper insights and trends discovery from the marks data.

\item \textbf{Continuous Model Improvement:} Regular updates and improvements to the underlying CNN model can further enhance the accuracy and performance of the application.

\item \textbf{Integration with Cloud Services:} Leveraging cloud services can provide scalability and accessibility, allowing users to access and process marks data from anywhere at any time.

\item \textbf{Mobile Application:} Developing a mobile application for Marks2CSV would enable educators and administrators to access and manage marks data on-the-go. This would provide flexibility and convenience, allowing them to efficiently perform tasks such as data entry, verification, and analysis directly from their mobile devices.
\end{enumerate}
These future scopes demonstrate the potential for the Marks2CSV application to evolve into a comprehensive and powerful tool for educational institutions, catering to their specific needs and driving further advancements in marks data management and analysis.

\clearpage

\section{Limitations}

This project aims to assist teachers and educational institutions in the efficient digitalization of handwritten marks on answer sheets. By implementing this system, users can benefit from significant time savings. However, it is important to acknowledge that the project does have certain limitations. The limitations are:
\begin{itemize}
    \item The system's inability to accurately detect decimal numbers, especially when they are represented using fractions, such as 0.5. This issue arises due to the different writing styles of teachers while writing decimal numbers (as in the above Figure 3.7) and the constraints of the algorithm employed for number detection.

    \item The efficiency of the Optical Character Recognition (OCR) tool used in the project. OCR tools are designed to convert scanned images that contains handwritten digit into machine-readable text, but they can be susceptible to errors in certain scenarios. Specifically, the tool's performance may be adversely affected by the presence of stray marks, corrections, or unsteady cell lines (as in the above Figure 3.9), which may result in inaccurate recognition or misinterpretation of the text.
    
    \item The impact of lighting conditions on the system's performance. In instances where images are captured under poor lighting conditions, such as low light or uneven illumination, the system may encounter difficulties in accurately detecting and extracting tables and marks from the images.

    \item Limitations in terms of scalability and compatibility with different formats and systems. The system may work well with specific types of answer sheets but may face difficulties in adapting to different formats used by various educational institutions. This can require additional customization or integration efforts to ensure compatibility across different systems.
\end{itemize} 

