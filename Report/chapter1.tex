\setcounter{equation}{0}
\chapter{Introduction}
The rapid advancement of Artificial Intelligence (AI) has significantly facilitated human workflow, particularly in the field of data processing. One prominent example is the digitalization of students' exam scores, which serves various purposes. This process is relatively simple for a small number of students, such as five or ten, . However, when dealing with an entire class or department, it becomes a challenging task that is both time-consuming and susceptible to errors.\\
To address this challenge, the objective of this project is to develop a reliable system capable of instantaneously converting answer sheet data into CSV files. By doing so, we aim to ease the considerable burden placed on our teachers during the manual data entry process. Consequently, this system will empower educators to allocate their valuable time more efficiently towards other essential responsibilities.

\setlength{\parskip}{2ex}

\section{Background}
\noindent In modern education, the digitalization of student exam scores is crucial for various administrative and academic purposes. At present, teachers are facing difficulties with the responsibility of manually entering student scores into spreadsheets or databases. This process involves precise attention to detail, repetitive data entry tasks, and a significant investment of time and effort. Moreover, the human element introduces the risk of transcription errors, leading to inaccurate data and potential variances in record-keeping.\\
However, recent advancements in AI and machine learning have presented opportunities to streamline and automate such labor-intensive processes. Utilizing AI algorithms, optical character recognition (OCR) techniques, and intelligent data processing, it is now possible to develop a system that can quickly and accurately extract relevant data from answer sheets and convert it into standardized CSV files.\\
By implementing an automated conversion system, educational institutions can achieve several benefits. Firstly, it significantly reduces the workload on teachers which result in freeing up their valuable time to focus on more critical tasks, such as curriculum development, student engagement, and individualized instruction. Also, an automated system minimizes the risk of manual data entry errors, ensuring the integrity and accuracy of student records.

\section{Motivation}
\noindent This project aims to address the challenges faced by teachers during the digitalization of student exam scores. The primary motivation behind this project is to utilize advancements in AI and machine learning to develop an automated system that can instantly convert answer sheet data into CSV files. By doing so, this project seeks to relieve educators from the difficulties of manual data entry and to improve efficiency in educational institutions. Implementing an automated system saves valuable time for teachers that allows them to focus on essential tasks such as lesson planning, student support and also professional development.



\section{Objective and Scope}
\subsection{Scope}

The scope of this project is to develop an automated system capable of converting images of answer sheets into CSV format. This will involve employing neural networks to accurately classify digits, ensuring optimal results. Additionally, the system will be designed to detect decimal marks during the OCR processing stage. A key aspect of the project is to train the model according to the specific requirements of the customers, allowing for customization in mark detection. Furthermore, a compact and portable device will be created to encapsulate the tool, enhancing usability and convenience.

\subsection{Objective}
The objective of this project is to develop a tool to assist teachers in the data entry process by implementing an efficient Optical Character Recognition (OCR) system using the Convolutional Neural Network (CNN) architecture. The tool aims to streamline the extraction of table marks from various images, enhancing accuracy and reducing manual effort. By utilizing the power of OCR technology, the tool will automate the recognition and extraction of digits from images which save teachers valuable time and minimize errors. Furthermore, the customizable nature of the program will enable programmers to define their specific parameters and rules for extracting marks as per institutional requirements. Ultimately, this project aims to provide teachers with a reliable and efficient solution for data entry, optimizing their productivity and facilitating their administrative tasks.

\section{Contributions}

This project aligns with the exploration and utilization of groundbreaking technologies for educational improvement. By developing an AI-based system for automated data entry processes, the project contributes to the integration of trending technologies within the education sector. It establishes a foundation for future advancements, fostering innovation in data management, analysis, and further integration of AI and machine learning in educational processes.






