\chapter{Literature Review}
 
%%%%%%%%%%%%%%%%%%%%%%%%%%%%%%%%%%%%%%%%%%%%%%%%%%%%%%%%%%%%
%%%%%%%%%%%%%%%%%%%%  NEW SECTION   %%%%%%%%%%%%%%%%%%%%%%%%
%%%%%%%%%%%%%%%%%%%%%%%%%%%%%%%%%%%%%%%%%%%%%%%%%%%%%%%%%%%%
\setcounter{equation}{0}
\section{System Description}
Donating blood is a safe, simple, and rewarding experience. When we arrive at a blood donation centre we will be asked to complete a donor registration form, which includes the name, address, phone number, and various other types of demographic information. He/she will also be asked to show the donor card or the type of identification required by the particular blood bank you visit.

\noindent Eligibility rules of the donor is to help protect the health and safety of the donor as well as the person who will receive a blood transfusion. Before donating, one of our medical professionals will discuss your health history with you in a private, confidential setting. After taking your pulse, blood pressure, and temperature and checking for anemia, we will determine whether you are eligible to be a donor. To donate blood or platelets, you must be in good general health, weigh at least 110 pounds, and be at least 16 years old. Parental consent is required for blood donation by 16 year old's, 16 year old's are not eligible to donate platelets. No parental consent is required for those who are at least 17 years old. If you are 76 or older, you will need your doctor’s written approval for blood or platelet donation. Blood donors must wait at least 56 days between blood donations and 7 days before donating platelets. Platelet donors may donate once every seven days, not to exceed six times in any eight-week period, and must wait 7 days before donating blood [3]. \par
\setlength{\parskip}{3ex}
\noindent
There are two types of blood donations- allogenic and directed. In 'allogenic' donation, a donor donates blood for storage in blood banks and this blood can be used for transfusion to any unknown recipient. In 'directed' donation, the donor is a relative or a family member of the recipient. There is a hybrid of these two called 'replacement donor' donation in which a friend or family member of the recipient donates blood to replace the stored blood used in transfusion, so that the availability of blood in blood banks is consistent. An event organized for people to donate blood is called a 'blood drive'.\par
\setlength{\parskip}{3ex}

%\subsection{Recipient and Donor Safety}
Firstly, a consent for donation is obtained from the donor. The donor has to go through a series of questionnaire and tests to detect whether there are any health risks that can make the donation unsafe for the recipient. Donors are examined for symptoms of diseases such as HIV, malaria, and viral hepatitis, ie, diseases that can be transmitted in a transfusion.

\noindent
If a potential donor does not meet the specified criteria for health, they are 'deferred', ie, he/she may be allowed to donate later when the health criteria is met. The donor is examined and asked specific questions about their medical history to make sure that the process of blood donation will not result in any hazardous effects on them. The donor's hemoglobin level is tested to make sure that the loss of blood will not make them anemic.
%\subsection{Blood Tests}
\noindent The donor's blood type is usually identified as type A, B, AB, or O and Rh type is also identified. Tests including a crossmatch is usually done before a transfusion process. Type O negative is referred to as the "universal donor" in the case of red cell and whole blood transfusions. For plasma and platelet transfusions, AB positive is the universal platelet donor while both AB positive and AB negative are universal plasma donors.\par
%\section{Existing Solutions}
\noindent
The blood is tested for diseases including STDs [4]. A variety of tests for infections transmitted during transmission is also conducted. Sometimes multiple tests may be used for the determination of a single disease. At each donation, the following mandatory tests are performed:

\begin{itemize}
\item Hepatitis B – HBsAg
\item Human immunodeficiency virus – anti-HIV 1 and 2 and HIV NAT (nucleic acid testing)\nomenclature{HIV}{Human Immunodeficiency Virus}
\item Hepatitis C – anti-HCV and HCV NAT
\item Human T-cell lymphotropic virus – anti-HTLV I and II
\item Syphilis – syphilis antibodies.
\item Some donations are tested for cytomegalovirus (CMV)\nomenclature{CMV}{cytomegalovirus} antibodies to provide CMV negative blood for patients with certain types of impaired immunity.
\end{itemize}

%\subsection{Blood Collection}
\noindent
Blood is obtained from the donor mainly in two ways:
\begin{itemize}
\item The simplest way is to take the blood from a vein as whole blood. This blood is separated into parts, usually red blood cells and plasma, since most recipients need only a specific component for transfusions.
\item The other method is to draw blood from the donor, separate it using a centrifuge or a filter, store the desired part, and return the rest to the donor. This process is called apheresis, and it is often done with a machine specifically designed for this purpose. This process is especially common for plasma and platelets.
\end{itemize}

%\subsection{Recovery and Time between Donations}
\noindent
Donors are observed for 10–15 minutes after donating since most reactions take place during or immediately after the donation. Blood centers typically provide refreshments or a lunch allowance to help the donor recover. The needle site is covered with a bandage and the donor is advised to keep the bandage on for some more time. Donors are also advised to avoid dehydration until a few hours after donation.

%\subsection{Storage and Blood Shelf Life}
\noindent
The collected blood is stored in the blood bank as separate components, and some of these components have short shelf lives. There are no storage methods to keep platelets for extended periods of time(no more than seven days). Red blood cells (RBC), which is the most frequently used component, have a shelf life of 35–42 days at refrigerated temperatures. Plasma can be stored the longest (upto one year) and maintaining its supply not a problem.



\section{Existing solutions}
There are some sources that provide an online platform for blood donation:
\subsection{American Red Cross Blood Services}
 The website is owned by American National Red Cross Society which is a well renowned organization for health services. [5] This website can be used by individuals who are willing to donate blood. They conduct blood drives to collect blood from donors and distribute it to the needed blood banks. They collaborate with various events like Superbowl to avail offers to the people donating blood. This website also gives provision to the user to conduct blood drives and we can also register to be part of their activities as a volunteer. But they do not provide the option to perform a emergency blood request even though that is vital part of the whole process.
% \subsection{Blood Bank India}
The website [6] provides various facilities like searching availability of blood, donor registration, and requesting blood. Latest requests are shown when one opens the website, the recent donors are also referred. The website does not provide accurate location based search results and hence it will not be a reliable source in every scenario. There is no integration with blood banks. Any random user can obtain the contact details of donors without any steps of verification, and legitimacy of donors are not verified. 
%\subsection{e-RaktKosh}
e-RaktKosh is a Centralized Blood Bank Management System [7]. It is an initiative of Ministry of Health and Family Welfare. It provides details about blood banks all across the nation. The details include the availability of each blood groups. But the information provided is not accurate. They also provide contact details and location information about blood banks.
\section{Summary}
