\chapter{Literature Review}
 
%%%%%%%%%%%%%%%%%%%%%%%%%%%%%%%%%%%%%%%%%%%%%%%%%%%%%%%%%%%%
%%%%%%%%%%%%%%%%%%%%  NEW SECTION   %%%%%%%%%%%%%%%%%%%%%%%%
%%%%%%%%%%%%%%%%%%%%%%%%%%%%%%%%%%%%%%%%%%%%%%%%%%%%%%%%%%%%
\setcounter{equation}{0}
\section{System Description}

Our system uses a camera to accept images as input as per our requirements. This obtained input is then processed using the \textbf{img2table} library. Using this library, the cell coordinates are extracted and cropped to obtain the values inside the cell. These values are then given to the TensorFlow OCR model to accurately classify and predict the values that we obtained from the cell extraction step. The obtained result from this step is arranged specially in the format of CSV files.

\section{Existing Solutions}
Handwritten number recognition has great importance in many fields such as educational, financial, and governmental institutions [1]. Many researchers have attempted to do the same using different techniques, like the neural network fusion method, clustering algorithm methods, or methods based on support vector machines (SVM) [11]. But the above-mentioned methods have performed below the expected target value and have also failed to meet the recognition accuracy that we aim for.

\noindent 
Based on the research done, we have found that convolutional neural networks have provided great image recognition results and avoided all complications found in the traditional recognition methods. Traditional CNNs adopt activation functions like Softmax to perform the classification and recognition processes. But with development in this field of study by combining CNN with SVM [10], they have greatly improved model performance. But the elementary factor here is that these works are performed on a system with advanced hardware.
\par
\setlength{\parskip}{3ex}
\noindent
The proposed network focuses mainly on four arithmetic operations - addition, deduction, multiplication, and division [6]. To correctly execute all of these operations, we need to take images with better quality, with the cells of the pictures aligned properly so that the captured image is not misclassified. It goes through many stages of image recognition procedures like skew image correction, image segmentation [11], number recognition, training data acquisition, assignments check, and algorithm improvement.

\par
\setlength{\parskip}{3ex}

%\subsection{LeNet5}
\noindent
LeNet-5 is used as the basic structure for handwritten data in this literature review. It consists of an input layer, convolutional layers, pooling layers, fully connected layers, and an output layer. LeNet-5 is utilized to extract image features and enhance the performance of Convolutional Neural Networks (CNNs) in recognizing handwritten data. And the model's performance is determined by the optimization algorithm used. [6]

%\subsection{Activation Functions}
\noindent 
Various activation functions are used and tested to check which helps provide better performance. Sigmoid and Tanh are exponential functions and hence require greater computation time than ReLU when calculating the error gradient of back-propagation. Another advantage is that it sets the final results of some neurons to be 0 which improves the network's sparsity, reduces overfitting possibilities, and the over-dependence on its parameters.
\par

%\section{Existing Solutions}
\noindent
Usage of the ADAM (Adaptive moment estimation) optimizer can eliminate the issue of rendering different training dynamic values on batches that undergo stochastic gradient descent during the data training procedure. ADAM optimizer is the equivalent of momentum and root mean square propagation ($RMS_{prop}$). $RMS_{prop}$  is an adaptive learning method that was proposed by Geoffrey E. Hinton, the godfather of AI. It can be used to avoid the continuous accumulation of second-order momentum and improve training speed with a larger learning rate.

%\subsection{Result and Analysis}
\noindent
CNNs achieve accurate classification for both printed and handwritten fonts [9]. Instead of manual number recognition, the model accurately identifies operations and consistently performs well in recognizing the questions. The model was tested on the MNIST dataset [13], which consisted of 60,000 training sample images and 10,000 testing sample images. After optimization, the CNN's recognition rate increased by 7.3\% to reach 91.2\%, and the convergence speed of the model decreased from 250 to 200, indicating improved accuracy.


\section{Summary}
\noindent
Thus we conclude that the CNN model with ReLU activation and ADAM optimizer is implemented on the MNIST dataset. The model is tested on four arithmetic operations - addition, deduction, multiplication, and division. The CNN model has also gained a reduction from 250 to 200 in convergence speed, and an increment from 83.9\% to 91.2\% in recognition accuracy.\\
Future work can be applied to datasets containing handwritten English and Chinese characters. The CNN model can potentially save the teacher's time in evaluating the handwritten documents and can devote more time towards creating learning activities that can help benefit the students.